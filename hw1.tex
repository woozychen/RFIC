\documentclass[12pt,a4paper]{article}
\usepackage{palatino}
\usepackage{geometry}
\geometry{left=1.25in,right=1.25in,top=1in,bottom=1in}


\begin{document}
\title{RFIC Homework 1}
\author{
Name: CHEN Li  \\Student ID: 5092119006
}

\date{2012/09/24}
\maketitle

\section*{Problem 1}
%1.a
\subsection*{(a)}
\textbf{Verify the above equation for at least three stages.}\\
\newline
For stage 1
\begin{equation}
N_1=G_1N_i+N_{n1}
\end{equation}
For stage 2
\begin{equation}
N_2=G_2N_1+N_{n2}
\end{equation}
For stage 3
\begin{equation}
N_3=G_3N_2+N_{n3}
\end{equation}
Combine the above 3 equations
\begin{equation}
N_3=G_1G_2G_3N_i+G_2G_3N_{n1}+G_3N_{n2}+N_{n3}
\end{equation}
If we view  the whole 3 stages as 1 stage
\begin{equation}
N_3=G_1G_2G_3N_i+N_{n123}
\end{equation}
Compare equation 5 with 4, we can derive that 
\begin{equation}
N_{n123}=G_2G_3N_{n1}+G_3N_{n2}+N_{n3}
\end{equation}
\newline
For stage 1
\begin{equation}
F_1=1+\frac{N_{n1}}{GNi}
\end{equation}
which is equivalent to
\begin{equation}
N_{n1}=(F_1-1)G_1N_i
\end{equation}
For stage 2
\begin{equation}
N_{n2}=(F_2-1)G_2N_i
\end{equation}
For stage 3
\begin{equation}
N_{n3}=(F_3-1)G_3N_i
\end{equation}
So if we view  the whole 3 stages as 1 stage
\begin{equation}
N_{n123}=(F_{total}-1)G_1G_2G_3N_i
\end{equation}
Combine equation 11 with 6
\begin{equation}
G_2G_3N_{n1}+G_3N_{n2}+N_{n3}=(F_{total}-1)G_1G_2G_3N_i
\end{equation}
Combine equation 12 with 8,9,10
\begin{equation}
G_2G_3(F_1-1)G_1N_i+G_3(F_2-1)G_2N_i+(F_3-1)G_3N_i=(F_{total}-1)G_1G_2G_3N_i
\end{equation}
divide both sides by $G_1G_2G_3N_i$ we will get
\begin{equation}
F_1-1+\frac{F_2-1}{G_1}+\frac{F_3-1}{G_1G_2}=F_{total}-1
\end{equation}
which is equivalent to
\begin{equation}
F_1+\frac{F_2-1}{G_1}+\frac{F_3-1}{G_1G_2}=F_{total}
\end{equation}
%1.b
\subsection*{(b)}
\textbf{Calculate the overall Noise Figure of the three cascaded amplifiers shown in Figure 1.}\\
\newline
\begin{eqnarray*}
F_{total}&=&F_1+\frac{F_2-1}{G_1}+\frac{F_3-1}{G_1G_2}\\
&=&10^{\frac{3}{10}}+\frac{10^{\frac{7}{10}}-1}{10^{\frac{7}{10}}}+\frac{10^{\frac{15}{10}}-1}{10^{\frac{7}{10}}*10^{\frac{10}{10}}}\\
&=&2.00+0.80+0.61\\
&=&3.41\\
&=&5.32 dB
\end{eqnarray*}
%1.c
\subsection*{(c)}
\textbf{If you would like to design an amplifier with a power gain of 50dB from two separate amplifiers, each with the characteristics depicted in Figure , find the optimal order and explain why.}\\
\newline
To achieve a low NF $F_{total}$, G1 should be larger but the dominant NF $F_1$ should be smaller.
\begin{equation}
F_{total}=F_1+\frac{F_2-1}{G_1}
\end{equation}
\subsection*{case 1}
stage 1 $G_1 F_1$ stage 2 $G_2 F_2$
\begin{eqnarray*}
F_{total1}&=&F_1+\frac{F_2-1}{G_1}\\
&=&10^{\frac{1.7}{10}}+\frac{10^{\frac{1.3}{10}}-1}{10^{\frac{30}{10}}}\\
&=&1.48\\
&=&1.70 dB
\end{eqnarray*}
\subsection*{case 2}
stage 1 $G_2 F_2$ stage 2 $G_1 F_1$
\begin{eqnarray*}
F_{total2}&=&F_2+\frac{F_1-1}{G_2}\\
&=&10^{\frac{1.3}{10}}+\frac{10^{\frac{1.7}{10}}-1}{10^{\frac{20}{10}}}\\
&=&1.35\\
&=&1.31 dB
\end{eqnarray*}
$F_{total1}>F_{total2}$, so the optimal order is stage 1 $G_2 F_2$ and stage 2 $G_1 F_1$
\section*{Problem 2}
\subsection*{$A_{1dB}$}
\begin{eqnarray*}
x(t)&=&A\cos{\omega{t}}\\
y(t)&=&a_1x(t)+a_2x^2(t)+a_3x^3(t)\\
&=&a_1A\cos{\omega{t}}+a_2A^2\cos^2{\omega{t}}+a_3A^3\cos^3{\omega{t}}
\end{eqnarray*}
the third-order part of output
\begin{eqnarray*}
a_3A^3\cos^3{\omega{t}}&=&a_3A^3\cos{\omega{t}}*\frac{1}{2}(1+\cos{2\omega{t}})\\
&=&a_3A^3(\frac{1}{2}\cos{\omega{t}}+\frac{1}{2}\cos{\omega{t}}*\cos{2\omega{t}})\\
&=&a_3A^3(\frac{1}{2}\cos{\omega{t}}+\frac{1}{4}\cos{\omega{t}}+\frac{1}{4}\cos{3\omega{t}})\\
&=&a_3A^3(\frac{3}{4}\cos{\omega{t}}+\frac{1}{4}\cos{2\omega{t}})
\end{eqnarray*}
the equivalent first-order part of output is $a_3A^3(\frac{3}{4}\cos{\omega{t}})(a_3<0)$\\
$A_{1dB}$ is the point when the nominal first-order part minus real first-order equals $1dB$
\begin{eqnarray*}
(a_1A\cos{\omega{t}}+\frac{3}{4}a_3A^3\cos{\omega{t}})-A\cos{\omega{t}}&=&-1dB\\
20\log{(1+\frac{3}{4}\frac{a_3}{a_4}A^2)}&=&-1dB
\end{eqnarray*}
thus
\begin{equation}
A_{1db}=\sqrt{\frac{4}{3}\left\vert{\frac{a_1}{a_3}}\right\vert}*\sqrt{0.109}
\end{equation}
\subsection*{$A_{IP3}$}
\begin{eqnarray*}
x(t)&=&A\cos{\omega_1{t}}+A\cos{\omega_2{t}}\\
y(t)&=&a_1x(t)+a_2x^2(t)+a_3x^3(t)
\end{eqnarray*}
the third-order part of output
\begin{eqnarray*}
a_3x^3(t)&=&a_3(A\cos{\omega_1{t}}+A\cos{\omega_2{t}})^3\\
&=&a_3(A^3\cos^3{\omega_1{t}}+A^3\cos^3{\omega_2{t}}+3A^3\cos^2{\omega_1{t}}\cos{\omega_2{t}}+3A^3\cos{\omega_1{t}}\cos^2{\omega_2{t}})
\end{eqnarray*}
only consider the $2\omega_1-\omega_2$ and $2\omega_2-\omega_1$ component 
\begin{eqnarray*}
3a_3A^3\cos^2{\omega_1{t}}\cos{\omega_2{t}}&=&\frac{3}{2}a_3A^3(1+\cos{2\omega_1{t}})\cos{\omega_2{t}}\\
&=&\frac{3}{2}a_3A^3(\cos{\omega_2{t}}+\frac{1}{2}(\cos{(2\omega_1+\omega_2)}+\cos{(2\omega_1-\omega_2)))}
\end{eqnarray*}
thus the $2\omega_1-\omega_2$ and $2\omega_2-\omega_1$ components are $\frac{3}{4}a_3A^3\cos{(2\omega_1+\omega_2)}$,  $\frac{3}{4}a_3A^3\cos{(2\omega_1-\omega_2)}$,  $\frac{3}{4}a_3A^3\cos{(2\omega_2+\omega_1)}$ and  $\frac{3}{4}a_3A^3\cos{(2\omega_2-\omega_1)}$\\
\newline
the $\omega_1$ and $\omega_2$ components are $a_1A\cos{\omega_1{t}}$ and $a_1A\cos{\omega_2{t}}$ \\
\newline
$A_{IP3}$ is the point when the $2\omega_1-\omega_2$ and $2\omega_2-\omega_1$ components equal the $\omega_1$ and $\omega_2$ components
\begin{equation}
\frac{3}{4}a_3A_{IP3}^3=a_1A_{IP3}
\end{equation}
thus
\begin{equation}
A_{IP3}=\sqrt{\frac{4}{3}\left\vert{\frac{a_1}{a_3}}\right\vert}
\end{equation}
thus the relation between $A_{1db}$ and $A_{IP3}$ is
\begin{equation}
A_{1db}=A_{IP3}*\sqrt{0.109}
\end{equation}
\end{document}